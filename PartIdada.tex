\subsection*{Amplicon bioinformatics: from raw reads to tables}
This section demonstrates
the ``full stack'' of amplicon bioinformatics: construction of the
sample-by-sequence feature table from the raw reads, assignment of
taxonomy, and creation of a phylogenetic tree relating the sample
sequences.

First we load the necessary packages.

\begin{knitrout}
\definecolor{shadecolor}{rgb}{0.969, 0.969, 0.969}\color{fgcolor}\begin{kframe}
\begin{alltt}
\hlkwd{library}\hlstd{(}\hlstr{"knitr"}\hlstd{)}
\hlkwd{library}\hlstd{(}\hlstr{"BiocStyle"}\hlstd{)}
\hlstd{opts_chunk}\hlopt{$}\hlkwd{set}\hlstd{(}\hlkwc{cache} \hlstd{=} \hlnum{FALSE}\hlstd{,}\hlkwc{fig.path}\hlstd{=}\hlstr{"dadafigure/"}\hlstd{)}
\hlkwd{read_chunk}\hlstd{(}\hlkwd{file.path}\hlstd{(}\hlstr{"src"}\hlstd{,} \hlstr{"bioinformatics.R"}\hlstd{))}
\end{alltt}
\end{kframe}
\end{knitrout}

\begin{knitrout}
\definecolor{shadecolor}{rgb}{0.969, 0.969, 0.969}\color{fgcolor}\begin{kframe}
\begin{alltt}
\hlstd{.cran_packages}  \hlkwb{<-}  \hlkwd{c}\hlstd{(}\hlstr{"ggplot2"}\hlstd{,} \hlstr{"gridExtra"}\hlstd{)}
\hlstd{.bioc_packages}  \hlkwb{<-} \hlkwd{c}\hlstd{(}\hlstr{"dada2"}\hlstd{,} \hlstr{"phyloseq"}\hlstd{,} \hlstr{"DECIPHER"}\hlstd{,} \hlstr{"phangorn"}\hlstd{)}

\hlstd{.inst} \hlkwb{<-} \hlstd{.cran_packages} \hlopt \hlkwd{installed.packages}\hlstd{()}
\hlkwa{if}\hlstd{(}\hlkwd{any}\hlstd{(}\hlopt{!}\hlstd{.inst)) \{}
  \hlkwd{install.packages}\hlstd{(.cran_packages[}\hlopt{!}\hlstd{.inst])}
\hlstd{\}}

\hlstd{.inst} \hlkwb{<-} \hlstd{.bioc_packages} \hlopt \hlkwd{installed.packages}\hlstd{()}
\hlkwa{if}\hlstd{(}\hlkwd{any}\hlstd{(}\hlopt{!}\hlstd{.inst)) \{}
  \hlkwd{source}\hlstd{(}\hlstr{"http://bioconductor.org/biocLite.R"}\hlstd{)}
  \hlkwd{biocLite}\hlstd{(.bioc_packages[}\hlopt{!}\hlstd{.inst],} \hlkwc{ask} \hlstd{= F)}
\hlstd{\}}

\hlcom{# Load packages into session, and print package version}
\hlkwd{sapply}\hlstd{(}\hlkwd{c}\hlstd{(.cran_packages, .bioc_packages), require,} \hlkwc{character.only} \hlstd{=} \hlnum{TRUE}\hlstd{)}
\end{alltt}

\begin{alltt}
\hlkwd{set.seed}\hlstd{(}\hlnum{100}\hlstd{)}

\hlcom{# you have to change this (unless we put the data on dropbox and}
\hlcom{# automatically download}
\end{alltt}
\end{kframe}
\end{knitrout}

The data we will analyze here are highly-overlapping Illumina Miseq
2x250 amplicon sequences from the V4 region of the 16S gene \cite{Kozich2013}.
These 360 fecal samples were collected from 12 mice longitudinally over the
first year of life, to investigate the development and
stabilization of the murine microbiome \cite{schloss2012stabilization}.
These data are downloaded from the following location:
\url{http://www.mothur.org/MiSeqDevelopmentData/StabilityNoMetaG.tar}.

\begin{knitrout}
\definecolor{shadecolor}{rgb}{0.969, 0.969, 0.969}\color{fgcolor}\begin{kframe}
\begin{alltt}
\hlstd{miseq_path} \hlkwb{<-} \hlkwd{file.path}\hlstd{(}\hlstr{"data"}\hlstd{,} \hlstr{"MiSeq_SOP"}\hlstd{)}
\hlstd{filt_path} \hlkwb{<-} \hlkwd{file.path}\hlstd{(}\hlstr{"data"}\hlstd{,} \hlstr{"filtered"}\hlstd{)}

\hlkwa{if}\hlstd{(}\hlopt{!}\hlkwd{file_test}\hlstd{(}\hlstr{"-d"}\hlstd{, miseq_path)) \{}
  \hlkwd{dir.create}\hlstd{(miseq_path)}
  \hlkwd{download.file}\hlstd{(}\hlstr{"http://www.mothur.org/MiSeqDevelopmentData/StabilityNoMetaG.tar"}\hlstd{,}
               \hlkwc{destfile} \hlstd{=} \hlkwd{file.path}\hlstd{(miseq_path,} \hlstr{"StabilityNoMetaG.tar"}\hlstd{))}
  \hlkwd{system}\hlstd{(}\hlkwd{paste0}\hlstd{(}\hlstr{"tar -xvf "}\hlstd{,} \hlkwd{file.path}\hlstd{(miseq_path,} \hlstr{"StabilityNoMetaG.tar"}\hlstd{),}
               \hlstr{" -C "}\hlstd{, miseq_path,} \hlstr{"/"}\hlstd{))}
\hlstd{\}}

\hlstd{fns} \hlkwb{<-} \hlkwd{sort}\hlstd{(}\hlkwd{list.files}\hlstd{(miseq_path,} \hlkwc{full.names} \hlstd{=} \hlnum{TRUE}\hlstd{))}
\hlstd{fnFs} \hlkwb{<-} \hlstd{fns[}\hlkwd{grepl}\hlstd{(}\hlstr{"R1"}\hlstd{, fns)]}
\hlstd{fnRs} \hlkwb{<-} \hlstd{fns[}\hlkwd{grepl}\hlstd{(}\hlstr{"R2"}\hlstd{, fns)]}
\end{alltt}
\end{kframe}
\end{knitrout}

\subsection*{Trim and Filter}

We begin by filtering out low-quality sequencing reads and trimming the
reads to a consistent length. While generally recommended filtering and
trimming parameters serve as a starting point, no two datasets are
identical and therefore it is always worth inspecting the quality of the
data before proceeding.

\begin{knitrout}
\definecolor{shadecolor}{rgb}{0.969, 0.969, 0.969}\color{fgcolor}\begin{kframe}
\begin{alltt}
\hlstd{ii} \hlkwb{<-} \hlkwd{sample}\hlstd{(}\hlkwd{length}\hlstd{(fnFs),} \hlnum{3}\hlstd{)}
\hlkwa{for}\hlstd{(i} \hlkwa{in} \hlstd{ii) \{} \hlkwd{print}\hlstd{(}\hlkwd{plotQualityProfile}\hlstd{(fnFs[i])} \hlopt{+} \hlkwd{ggtitle}\hlstd{(}\hlstr{"Fwd"}\hlstd{)) \}}
\hlkwa{for}\hlstd{(i} \hlkwa{in} \hlstd{ii) \{} \hlkwd{print}\hlstd{(}\hlkwd{plotQualityProfile}\hlstd{(fnRs[i])} \hlopt{+} \hlkwd{ggtitle}\hlstd{(}\hlstr{"Rev"}\hlstd{)) \}}
\end{alltt}
\end{kframe}
\end{knitrout}
\begin{figure}
\includegraphics[width=0.3\maxwidth]{dadafigure/profile-1}
\includegraphics[width=0.3\maxwidth]{dadafigure/profile-2}
\includegraphics[width=0.3\maxwidth]{dadafigure/profile-3}
\\
\includegraphics[width=0.3\maxwidth]{dadafigure/profile-4}
\includegraphics[width=0.3\maxwidth]{dadafigure/profile-5}
\includegraphics[width=0.3\maxwidth]{dadafigure/profile-6}
\caption{Forward and Reverse Error Profiles, the mean is in green, the median the solid orange line
and the quartiles are the dotted orange lines.}
\label{fig:r profile1}
\end{figure}

Most Illumina sequencing data shows a trend of decreasing average quality
towards the end of sequencing reads.

Here, the forward reads maintain high quality throughout, while the quality
of the reverse reads drops significantly at about position 160. Therefore,
we choose to truncate the forward reads at position 245, and the
reverse reads at position 160. We also choose to trim the first 10 nucleotides
of each read based on empirical observations across many Illumina datasets that
these base positions are particularly likely to contain pathological errors.

We combine these trimming parameters with standard filtering
parameters, the most important being the enforcement of a maximum of 2 expected
errors per-read \cite{edgar2015unoise}. Trimming and filtering is performed on
paired reads jointly, i.e. both reads must pass the filter for the pair to pass.

\begin{knitrout}
\definecolor{shadecolor}{rgb}{0.969, 0.969, 0.969}\color{fgcolor}\begin{kframe}
\begin{alltt}
\hlkwa{if}\hlstd{(}\hlopt{!}\hlkwd{file_test}\hlstd{(}\hlstr{"-d"}\hlstd{, filt_path))} \hlkwd{dir.create}\hlstd{(filt_path)}
\hlstd{filtFs} \hlkwb{<-} \hlkwd{file.path}\hlstd{(filt_path,} \hlkwd{basename}\hlstd{(fnFs))}
\hlstd{filtRs} \hlkwb{<-} \hlkwd{file.path}\hlstd{(filt_path,} \hlkwd{basename}\hlstd{(fnRs))}
\hlkwa{for}\hlstd{(i} \hlkwa{in} \hlkwd{seq_along}\hlstd{(fnFs)) \{}
  \hlkwd{fastqPairedFilter}\hlstd{(}\hlkwd{c}\hlstd{(fnFs[[i]], fnRs[[i]]),}
                    \hlkwd{c}\hlstd{(filtFs[[i]], filtRs[[i]]),}
                    \hlkwc{trimLeft}\hlstd{=}\hlnum{10}\hlstd{,} \hlkwc{truncLen}\hlstd{=}\hlkwd{c}\hlstd{(}\hlnum{245}\hlstd{,} \hlnum{160}\hlstd{),}
                    \hlkwc{maxN}\hlstd{=}\hlnum{0}\hlstd{,} \hlkwc{maxEE}\hlstd{=}\hlnum{2}\hlstd{,} \hlkwc{truncQ}\hlstd{=}\hlnum{2}\hlstd{,}
                    \hlkwc{compress}\hlstd{=}\hlnum{TRUE}\hlstd{)}
\hlstd{\}}
\end{alltt}
\end{kframe}
\end{knitrout}

\subsection*{Infer sequence variants}
After filtering, the typical amplicon bioinformatics workflow clusters
sequencing reads into operational taxonomic units (OTUs): groups of
sequencing reads that differ by less than a fixed dissimilarity
threshhold. Here we instead use the high-resolution DADA2
method to to infer ribosomal sequence variants (RSVs) exactly, without imposing
any arbitrary threshhold, and thereby
resolving variants that differ by as little as one nucleotide \cite{dada2}.

The sequence data is imported into R from demultiplexed fastq files (i.e. one
fastq for each sample) and simultaneously dereplicated to remove redundancy. We
name the resulting `derep-class` objects by their sample name.

\begin{knitrout}
\definecolor{shadecolor}{rgb}{0.969, 0.969, 0.969}\color{fgcolor}\begin{kframe}
\begin{alltt}
\hlstd{derepFs} \hlkwb{<-} \hlkwd{derepFastq}\hlstd{(filtFs)}
\hlstd{derepRs} \hlkwb{<-} \hlkwd{derepFastq}\hlstd{(filtRs)}
\hlstd{sam.names} \hlkwb{<-} \hlkwd{sapply}\hlstd{(}\hlkwd{strsplit}\hlstd{(}\hlkwd{basename}\hlstd{(filtFs),} \hlstr{"_"}\hlstd{), `[`,} \hlnum{1}\hlstd{)}
\hlkwd{names}\hlstd{(derepFs)} \hlkwb{<-} \hlstd{sam.names}
\hlkwd{names}\hlstd{(derepRs)} \hlkwb{<-} \hlstd{sam.names}
\end{alltt}
\end{kframe}
\end{knitrout}

The DADA2 method relies on a parameterized model of substitution
errors to distinguish sequencing errors from real biological
variation. Because error rates can (and often do) vary substantially
between sequencing runs and PCR protocols, the model parameters can be
discovered from the data itself using a form of unsupervised learning
in which sample inference is alternated with parameter estimation
until both are jointly consistent.

Parameter learning is computationally intensive, as it requires
multiple iterations of the sequence inference algorithm, and therefore
it is often useful to estimate the error rates from a (sufficiently
large) subset of the data.

\begin{knitrout}
\definecolor{shadecolor}{rgb}{0.969, 0.969, 0.969}\color{fgcolor}\begin{kframe}
\begin{alltt}
\hlstd{ddF} \hlkwb{<-} \hlkwd{dada}\hlstd{(derepFs[}\hlnum{1}\hlopt{:}\hlnum{40}\hlstd{],} \hlkwc{err}\hlstd{=}\hlkwa{NULL}\hlstd{,} \hlkwc{selfConsist}\hlstd{=}\hlnum{TRUE}\hlstd{)}
\end{alltt}


{\ttfamily\noindent\itshape\color{messagecolor}{\#\# Initial error matrix unspecified. Error rates will be initialized to the maximum possible estimate from this data.}}\begin{verbatim}
## Initializing error rates to maximum possible estimate.
## Sample 1 - 7084 reads in 1955 unique sequences.
## ..........
## Sample 40 - 4191 reads in 922 unique sequences.
## Convergence after  5  rounds.
\end{verbatim}
\begin{alltt}
\hlstd{ddR} \hlkwb{<-} \hlkwd{dada}\hlstd{(derepRs[}\hlnum{1}\hlopt{:}\hlnum{40}\hlstd{],} \hlkwc{err}\hlstd{=}\hlkwa{NULL}\hlstd{,} \hlkwc{selfConsist}\hlstd{=}\hlnum{TRUE}\hlstd{)}
\end{alltt}


{\ttfamily\noindent\itshape\color{messagecolor}{\#\# Initial error matrix unspecified. Error rates will be initialized to the maximum possible estimate from this data.}}\begin{verbatim}
## Initializing error rates to maximum possible estimate.
## Sample 1 - 7084 reads in 1548 unique sequences.
## .....................
## Sample 40 - 4191 reads in 999 unique sequences.
## Convergence after  6  rounds.
\end{verbatim}
\end{kframe}
\end{knitrout}

In order to verify that the error rates have been reasonably
well-estimated, we inspect the fit between the observed error rates (black
points) and the fitted error rates (black lines) in Figure \ref{fig:errorprofile1}.


\begin{knitrout}
\definecolor{shadecolor}{rgb}{0.969, 0.969, 0.969}\color{fgcolor}\begin{kframe}
\begin{alltt}
\hlkwd{plotErrors}\hlstd{(ddF)}
\hlkwd{plotErrors}\hlstd{(ddR)}
\end{alltt}
\end{kframe}
\end{knitrout}
\begin{figure}
\includegraphics[width=0.46\maxwidth]{dadafigure/plot-rates-1}
\includegraphics[width=0.46\maxwidth]{dadafigure/plot-rates-2}
\caption{Forward and Reverse Read Error Profiles, showing the
  frequencies of each type of nucleotide transition as a function of quality.}\label{fig:r
  errorprofile1}
\end{figure}

The DADA2 sequence inference method can run in two different modes:
Independent inference by sample (`pool=FALSE`), and inference from the
pooled sequencing reads from all samples (`pool=TRUE`). Independent
inference has the advantage that computation time is linear
in the number of samples, and memory requirements are flat with the
number of samples. This allows scaling out to datasets of almost unlimited
size. Pooled inference is more computationally taxing, and can become
intractable for datasets of tens of millions of reads.
However, pooling improves the detection of rare variants that
were seen just once or twice in an individual sample but many times across
all samples. As this dataset is not particularly large, we perform pooled
inference. *Of potential interest, the
\href{https://github.com/benjjneb/dada2}{development version of dada2}
has added multithreading, which can substantially speed this step.*

\begin{knitrout}
\definecolor{shadecolor}{rgb}{0.969, 0.969, 0.969}\color{fgcolor}\begin{kframe}
\begin{alltt}
\hlstd{dadaFs} \hlkwb{<-} \hlkwd{dada}\hlstd{(derepFs,} \hlkwc{err}\hlstd{=ddF[[}\hlnum{1}\hlstd{]]}\hlopt{$}\hlstd{err_out,} \hlkwc{pool}\hlstd{=}\hlnum{TRUE}\hlstd{)}
\end{alltt}
\begin{verbatim}
## 362 samples were pooled: 3342527 reads in 272916 unique sequences.
\end{verbatim}
\begin{alltt}
\hlstd{dadaRs} \hlkwb{<-} \hlkwd{dada}\hlstd{(derepRs,} \hlkwc{err}\hlstd{=ddR[[}\hlnum{1}\hlstd{]]}\hlopt{$}\hlstd{err_out,} \hlkwc{pool}\hlstd{=}\hlnum{TRUE}\hlstd{)}
\end{alltt}
\begin{verbatim}
## 362 samples were pooled: 3342527 reads in 278172 unique sequences.
\end{verbatim}
\end{kframe}
\end{knitrout}

The DADA2 sequence inference step removed (nearly) all substitution and
indel errors from the data \cite{dada2}. We now merge together the inferred forward
and reverse sequences, removing paired sequences that do not perfectly
overlap as a final control against residual errors.

\begin{knitrout}
\definecolor{shadecolor}{rgb}{0.969, 0.969, 0.969}\color{fgcolor}\begin{kframe}
\begin{alltt}
\hlstd{mergers} \hlkwb{<-} \hlkwd{mergePairs}\hlstd{(dadaFs, derepFs, dadaRs, derepRs)}
\end{alltt}
\end{kframe}
\end{knitrout}

\subsection*{Construct sequence table and remove chimeras}

The DADA2 method produces a sequence table that is a higher-resolution
analogue of the common "OTU table", i.e. a sample by sequence feature
table valued by the number of times each sequence was observed in each sample.

\begin{knitrout}
\definecolor{shadecolor}{rgb}{0.969, 0.969, 0.969}\color{fgcolor}\begin{kframe}
\begin{alltt}
\hlstd{seqtab.all} \hlkwb{<-} \hlkwd{makeSequenceTable}\hlstd{(mergers[}\hlopt{!}\hlkwd{grepl}\hlstd{(}\hlstr{"Mock"}\hlstd{,} \hlkwd{names}\hlstd{(mergers))])}
\end{alltt}
\end{kframe}
\end{knitrout}

Notably, chimeras have not yet been removed. The error model in the
sequence inference algorithm does not include a chimera component, and
therefore we expect this sequence table to include many chimeric sequences.
We now remove chimeric sequences by comparing each inferred sequence to the
others in the table, and removing those that can be reproduced by
stitching together two more abundant sequences.

\begin{knitrout}
\definecolor{shadecolor}{rgb}{0.969, 0.969, 0.969}\color{fgcolor}\begin{kframe}
\begin{alltt}
\hlstd{seqtab} \hlkwb{<-} \hlkwd{removeBimeraDenovo}\hlstd{(seqtab.all)}
\end{alltt}
\end{kframe}
\end{knitrout}

Although exact numbers vary substantially by experimental condition, it is
typical that chimeras comprise a substantial fraction of inferred sequence variants,
but only a small fraction of all reads. That is
what is observed here: 1503 of 1892 sequence variants were chimeric, but
these only represented 10\% of all reads.

\subsection*{Assign taxonomy}

One of the benefits of using well-classified marker loci like the 16S
rRNA gene is the ability to taxonomically classify the sequence
variants. The \href{http://bioconductor.org/packages/dada2}{dada2}
package implements the naive Bayesian classifier method for this
purpose \cite{wang2007naive}. This classifier compares sequence
variants to a training set of classified sequences, and here we use the
RDP v14 training set \cite{cole2009rdp}.

\begin{knitrout}
\definecolor{shadecolor}{rgb}{0.969, 0.969, 0.969}\color{fgcolor}\begin{kframe}
\begin{alltt}
\hlstd{ref_fasta} \hlkwb{<-} \hlstr{"data/rdp_train_set_14.fa.gz"}
\hlstd{taxtab} \hlkwb{<-} \hlkwd{assignTaxonomy}\hlstd{(seqtab,} \hlkwc{refFasta} \hlstd{= ref_fasta)}
\hlkwd{colnames}\hlstd{(taxtab)} \hlkwb{<-} \hlkwd{c}\hlstd{(}\hlstr{"Kingdom"}\hlstd{,} \hlstr{"Phylum"}\hlstd{,} \hlstr{"Class"}\hlstd{,} \hlstr{"Order"}\hlstd{,} \hlstr{"Family"}\hlstd{,} \hlstr{"Genus"}\hlstd{)}
\end{alltt}
\end{kframe}
\end{knitrout}

GreenGenes and Silva training set {\tt fasta} files formatted for the
`assignTaxonomy` function are also available for download at
\url{https://www.dropbox.com/sh/mfcivbudmc21cqt/AAB1l-AUM5uKvjrR33ct-cTXa?dl=0}.

\subsection*{Construct phylogenetic tree}
Phylogenetic relatedness is commonly used to inform downstream
analyses, especially the calculation of phylogeny-aware distances
between microbial communities. The DADA2 sequence inference method is
reference-free, so we must construct the phylogenetic tree relating
the inferred sequence variants de novo. We begin by performing a
multiple-alignment using the {\tt DECIPHER} R package
\cite{wright2015decipher}.

\begin{knitrout}
\definecolor{shadecolor}{rgb}{0.969, 0.969, 0.969}\color{fgcolor}\begin{kframe}
\begin{alltt}
\hlstd{seqs} \hlkwb{<-} \hlkwd{getSequences}\hlstd{(seqtab)}
\hlkwd{names}\hlstd{(seqs)} \hlkwb{<-} \hlstd{seqs} \hlcom{# This propagates to the tip labels of the tree}
\hlstd{alignment} \hlkwb{<-} \hlkwd{AlignSeqs}\hlstd{(}\hlkwd{DNAStringSet}\hlstd{(seqs),} \hlkwc{anchor}\hlstd{=}\hlnum{NA}\hlstd{)}
\end{alltt}
\end{kframe}
\end{knitrout}

The \CRANpkg{phangorn} R package is then used to construct a
phylogenetic tree. Here we first construct a neighbor-joining tree,
and then fit a GTR+G+I (Generalized time-reversible with Gamma rate variation) maximum likelihood tree using the
neighbor-joining tree as a starting point.

\begin{knitrout}
\definecolor{shadecolor}{rgb}{0.969, 0.969, 0.969}\color{fgcolor}\begin{kframe}
\begin{alltt}
\hlstd{phang.align} \hlkwb{<-} \hlkwd{phyDat}\hlstd{(}\hlkwd{as}\hlstd{(alignment,} \hlstr{"matrix"}\hlstd{),} \hlkwc{type}\hlstd{=}\hlstr{"DNA"}\hlstd{)}
\hlstd{dm} \hlkwb{<-} \hlkwd{dist.ml}\hlstd{(phang.align)}
\hlstd{treeNJ} \hlkwb{<-} \hlkwd{NJ}\hlstd{(dm)} \hlcom{# Note, tip order != sequence order}
\hlstd{fit} \hlkwb{=} \hlkwd{pml}\hlstd{(treeNJ,} \hlkwc{data}\hlstd{=phang.align)}
\end{alltt}


{\ttfamily\noindent\itshape\color{messagecolor}{\#\# negative edges length changed to 0!}}\begin{alltt}
\hlstd{fitGTR} \hlkwb{<-} \hlkwd{update}\hlstd{(fit,} \hlkwc{k}\hlstd{=}\hlnum{4}\hlstd{,} \hlkwc{inv}\hlstd{=}\hlnum{0.2}\hlstd{)}
\hlstd{fitGTR} \hlkwb{<-} \hlkwd{optim.pml}\hlstd{(fitGTR,} \hlkwc{model}\hlstd{=}\hlstr{"GTR"}\hlstd{,} \hlkwc{optInv}\hlstd{=}\hlnum{TRUE}\hlstd{,} \hlkwc{optGamma}\hlstd{=}\hlnum{TRUE}\hlstd{,}
                    \hlkwc{rearrangement} \hlstd{=} \hlstr{"stochastic"}\hlstd{,} \hlkwc{control} \hlstd{=} \hlkwd{pml.control}\hlstd{(}\hlkwc{trace} \hlstd{=} \hlnum{0}\hlstd{))}
\hlkwd{detach}\hlstd{(}\hlstr{"package:phangorn"}\hlstd{,} \hlkwc{unload}\hlstd{=}\hlnum{TRUE}\hlstd{)}
\end{alltt}
\end{kframe}
\end{knitrout}

\subsection*{Combine data into a phyloseq object}

The \BioCpkg{phyloseq} package
organizes and synthesizes the different data types from a typical
amplicon sequencing experiment into a single data object that can be
easily manipulated. The last bit of information needed is the sample
data contained in a \texttt{.csv} file.

\begin{knitrout}
\definecolor{shadecolor}{rgb}{0.969, 0.969, 0.969}\color{fgcolor}\begin{kframe}
\begin{alltt}
\hlstd{mimarks_path} \hlkwb{<-} \hlstr{"data/MIMARKS_Data_combined.csv"}
\hlstd{samdf} \hlkwb{<-} \hlkwd{read.csv}\hlstd{(mimarks_path,} \hlkwc{header}\hlstd{=}\hlnum{TRUE}\hlstd{)}
\hlstd{samdf}\hlopt{$}\hlstd{SampleID} \hlkwb{<-} \hlkwd{paste0}\hlstd{(}\hlkwd{gsub}\hlstd{(}\hlstr{"00"}\hlstd{,} \hlstr{""}\hlstd{, samdf}\hlopt{$}\hlstd{host_subject_id),} \hlstr{"D"}\hlstd{, samdf}\hlopt{$}\hlstd{age}\hlopt{-}\hlnum{21}\hlstd{)}
\hlstd{samdf} \hlkwb{<-} \hlstd{samdf[}\hlopt{!}\hlkwd{duplicated}\hlstd{(samdf}\hlopt{$}\hlstd{SampleID),]} \hlcom{# Remove duplicate entries for reverse reads}
\hlkwd{rownames}\hlstd{(seqtab)} \hlkwb{<-} \hlkwd{gsub}\hlstd{(}\hlstr{"124"}\hlstd{,} \hlstr{"125"}\hlstd{,} \hlkwd{rownames}\hlstd{(seqtab))} \hlcom{# Fixing an odd discrepancy}
\hlkwd{all}\hlstd{(}\hlkwd{rownames}\hlstd{(seqtab)} \hlopt \hlstd{samdf}\hlopt{$}\hlstd{SampleID)} \hlcom{# TRUE}
\end{alltt}
\begin{verbatim}
## [1] TRUE
\end{verbatim}
\begin{alltt}
\hlkwd{rownames}\hlstd{(samdf)} \hlkwb{<-} \hlstd{samdf}\hlopt{$}\hlstd{SampleID}
\hlstd{keep.cols} \hlkwb{<-} \hlkwd{c}\hlstd{(}\hlstr{"collection_date"}\hlstd{,} \hlstr{"biome"}\hlstd{,} \hlstr{"target_gene"}\hlstd{,} \hlstr{"target_subfragment"}\hlstd{,}
\hlstr{"host_common_name"}\hlstd{,} \hlstr{"host_subject_id"}\hlstd{,} \hlstr{"age"}\hlstd{,} \hlstr{"sex"}\hlstd{,} \hlstr{"body_product"}\hlstd{,} \hlstr{"tot_mass"}\hlstd{,}
\hlstr{"diet"}\hlstd{,} \hlstr{"family_relationship"}\hlstd{,} \hlstr{"genotype"}\hlstd{,} \hlstr{"SampleID"}\hlstd{)}
\hlstd{samdf} \hlkwb{<-} \hlstd{samdf[}\hlkwd{rownames}\hlstd{(seqtab), keep.cols]}
\end{alltt}
\end{kframe}
\end{knitrout}

The full suite of data for this study -- the sample-by-sequence
feature table, the sample metadata, the sequence taxonomies, and the
phylogenetic tree -- can now be combined into a single object.

\begin{knitrout}
\definecolor{shadecolor}{rgb}{0.969, 0.969, 0.969}\color{fgcolor}\begin{kframe}
\begin{alltt}
\hlstd{ps} \hlkwb{<-} \hlkwd{phyloseq}\hlstd{(}\hlkwd{tax_table}\hlstd{(taxtab),} \hlkwd{sample_data}\hlstd{(samdf),}
               \hlkwd{otu_table}\hlstd{(seqtab,} \hlkwc{taxa_are_rows} \hlstd{=} \hlnum{FALSE}\hlstd{),} \hlkwd{phy_tree}\hlstd{(fitGTR}\hlopt{$}\hlstd{tree))}
\end{alltt}
\end{kframe}
\end{knitrout}
